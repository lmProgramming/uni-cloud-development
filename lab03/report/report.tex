\documentclass[a4paper,12pt]{article}
\usepackage[utf8]{inputenc}
\usepackage[T1]{fontenc}
\usepackage{graphicx}
\usepackage{hyperref}
\usepackage{listings}
\usepackage{amsmath}
\usepackage{float}
\usepackage{geometry}
\usepackage{polski}
\usepackage{color}

\geometry{margin=1in}

\title{Sprawozdanie z laboratorium 3}
\author{Mikołaj Kubś 272662}
\date{\today}

\begin{document}

\maketitle

\section{Cel zadania}

Celem zadania jest stworzenie prostej aplikacja PWA z wykorzystaniem HTML, CSS i JS. Kolejnym krokiem był deployment i testy funkcjonalności.

\subsection{Wprowadzenie}

Progressive Web App to aplikacja internetowa uruchamiana tak jak zwykła strona internetowa, ale
umożliwiająca stworzenie wrażenia działania jak natywna aplikacja mobilna (lub aplikacja
desktopowa).

\subsection{Tworzenie aplikacja PWA z wykorzystaniem HTML, JS i CSS}

\subsubsection{Kod HTML}
Plik \texttt{index.html} definiuje prosty kod HTML aplikacji wraz z ikonami dla PWA i innymi informacjami:

\begin{figure}[H]
    \centering
    \includegraphics[width=1\textwidth]{images/index_html.png}
    \caption{Kod index.html}
\end{figure}

\subsubsection{Stylizacja CSS}
Plik \texttt{style.css} definiuje kaskadowe arkusze stylów:

\begin{figure}[H]
    \centering
    \includegraphics[width=1\textwidth]{images/css.png}
    \caption{Fragment kodu style.css}
\end{figure}

\subsubsection{Kod JavaScript}

\begin{figure}[H]
    \centering
    \includegraphics[width=1\textwidth]{images/main_js.png}
    \caption{Fragment kodu main.js}
\end{figure}

\begin{figure}[H]
    \centering
    \includegraphics[width=1\textwidth]{images/sw_js.png}
    \caption{Fragment kodu sw.js - service worker}
\end{figure}

\subsection{Dalsza implementacja}

Dzięki wykorzystaniu kodu w instrukcji implementacja przebiegła bez większych problemów. Najwięcej problemów wynikło ze ścieżek, gdyż publikując stronę w GitHub Pages okazuje się, że jest głębiej w folderze.

Do ikon wykorzystano oba podane narzędzia: \href{https://www.npmjs.com/package/pwa-asset-generator}{pwa-asset-generator} i \href{https://realfavicongenerator.net/}{RealFaviconGenerator}.

Rozbudowano stronę, dodając zdjęcia psów i podstronę ze zdjęciami kotów.

\begin{figure}[H]
    \centering
    \includegraphics[width=1\textwidth]{images/lighthouse.png}
    \caption{Wynik analizy Lighthouse}
\end{figure}

\begin{figure}[H]
    \centering
    \includegraphics[width=1\textwidth]{images/splash.jpg}
    \caption{Splash image}
\end{figure}

Service worker działa jak należy, cache'ując odpowiednie pliki. Po pobraniu aplikacji na telefon splash image się pokazuje.

Wdrożono aplikację w GitHub pages, co, jak wyżej wspomniano, wymagało naprawy ścieżek.

\subsection{Powiadomienia}

Sprawiły one najwięcej problemów implementacyjnych. Stworzono projekt w Firebase i skrypt \texttt{firebase-messaging-sw.js}. Po poprawie ścieżek w końcu zadziałało i dzięki wydrukowaniu tokenu FCM można było wysłać testową wiadomość z konsole Firebase.

\begin{figure}[H]
    \centering
    \includegraphics[width=1\textwidth]{images/notification.png}
    \caption{Powiadomienie}
\end{figure}

\end{document}
