\documentclass[a4paper,12pt]{article}
\usepackage[utf8]{inputenc}
\usepackage[T1]{fontenc}
\usepackage{graphicx}
\usepackage{hyperref}
\usepackage{listings}
\usepackage{amsmath}
\usepackage{float}
\usepackage{geometry}
\usepackage{polski}
\usepackage{color}

\geometry{margin=1in}

\title{Sprawozdanie z laboratorium 1}
\author{Mikołaj Kubś 272662}
\date{\today}

\begin{document}

\maketitle

\section{Cel zadania}
Celem zadania było zapoznanie się z procesem tworzenia prostych stron Single Page Application (SPA), a także integracją z Azure Static Web i Github Pages. Dodatkowo wykorzystano technikę leniwego ładowania i mechanizm reCAPTCHA.

\subsection{Wprowadzenie}
Single Page Application (SPA) to rodzaj aplikacji internetowej, w której nawigacja odbywa się poprzez asynchroniczne ładowanie poszczególnych elementów strony, takich jak sekcje lub całe widoki, bez konieczności przeładowywania całej strony. Dzięki temu użytkownik doświadcza płynniejszej interakcji, ponieważ zmiany w interfejsie są natychmiastowe i nie wymagają pełnego odświeżenia przeglądarki.

W aplikacjach SPA cała zawartość jest zazwyczaj ładowana jednorazowo przy pierwszym wejściu na stronę, a późniejsze interakcje z użytkownikiem prowadzą do dynamicznej aktualizacji wyświetlanych danych za pomocą JavaScript. To podejście pozwala na szybsze działanie aplikacji oraz lepsze wykorzystanie zasobów sieciowych, ponieważ jedynie zmieniane elementy są przesyłane między serwerem a klientem.

\section{Wykorzystane technologie}
\begin{itemize}
    \item jQuery
    \item Azure Static Web
    \item Github Pages
\end{itemize}

\subsection{Tworzenie aplikacji SPA z wykorzystaniem HTML i JS}

\subsubsection{Kod HTML}
Plik \texttt{index.html} definiuje prosty kod HTML aplikacji:


\subsubsection{Stylizacja CSS}
Plik \texttt{style.css} definiuje kaskadowe arkusze stylów:


\subsubsection{Kod JavaScript}
Plik \texttt{router.js} zarządza nawigacją w aplikacji SPA:


\subsection{Wdrożenie aplikacji w środowisku chmurowym}
\subsubsection{Azure Static Web Apps}
Azure Static Web Apps oferuje darmową usługę hostingową dla aplikacji SPA. Kroki wdrożenia:
\begin{enumerate}
    \item Zaloguj się na \url{https://portal.azure.com}.
    \item Utwórz nową aplikację statyczną, podając szczegóły projektu.
    \item Połącz aplikację z repozytorium GitHub.
    \item Wdróż aplikację i sprawdź jej działanie.
\end{enumerate}

\subsubsection{GitHub Pages}
GitHub Pages umożliwia hostowanie aplikacji SPA. Kroki wdrożenia:
\begin{enumerate}
    \item Utwórz repozytorium na GitHubie.
    \item Skonfiguruj GitHub Pages w ustawieniach repozytorium.
    \item Sprawdź dostępność aplikacji pod adresem \texttt{https://username.github.io/repository}.
\end{enumerate}

\subsection{Testowanie aplikacji}
Przeprowadź testy funkcjonalności aplikacji:
\begin{itemize}
    \item Sprawdź nawigację między stronami.
    \item Zweryfikuj poprawność ładowania obrazów w galerii.
    \item Przetestuj walidację formularza kontaktowego.
    \item Użyj narzędzia Lighthouse w ChromeDevTools do analizy wydajności.
\end{itemize}

\end{document}
