\documentclass[a4paper,12pt]{article}
\usepackage[utf8]{inputenc}
\usepackage[T1]{fontenc}
\usepackage{graphicx}
\usepackage{hyperref}
\usepackage{listings}
\usepackage{amsmath}
\usepackage{geometry}
\geometry{margin=1in}

\title{Sprawozdanie z laboratorium 1}
\author{Mikołaj Kubś 272662}
\date{}

\begin{document}

\maketitle

\section{Cel zadania}
Celem zadania było zapoznanie się z procesem tworzenia serwerów i stron internetowych za pomocą Node.js oraz Express.js. W późniejszych etapach należało utworzyć obraz Dockerowy, aby zbudować kontener z aplikacją. Ostatecznie należało przekształcić aplikację, aby działała w trybie serverless i wdrożyć ją na jednej z platform chmurowych.

\section{Wykorzystane technologie}
\begin{itemize}
    \item Node.js
    \item Express.js (wraz z Expressjs-layouts)
    \item serverless, serverless-http
    \item Docker - do tworzenia kontenerów
    \item Vercel - do hostowania aplikacji
\end{itemize}

Dodatkowo wszystkie pliki źródłowe zostały napisane w TypeScript, a proces kompilacji został zautomatyzowany przed każdym uruchomieniem aplikacji.

\section{Opis aplikacji}
Aplikacja jest stroną internetową typu Portfolio, która może zawierać informacje o użytkowniku, jego projektach oraz galerię zdjęć. Dotyczy konkretnie autora sprawozdania i jego osiągnięć. Dodatkowo aplikacja zawiera formularz kontaktowy, który w przyszłości umożliwi kontakt z autorem.

\section{Opis procesu}
\subsection{Tworzenie aplikacji}
W celu utworzenia aplikacji został stworzony plik o nazwie \texttt{server.ts}, który zawiera kod odpowiedzialny za uruchomienie serwera, konfigurację ścieżek oraz przekierowywanie za pomocą middleware.

Formularz kontaktowy jest obsługiwany przez middleware \texttt{body-parser}, który przekształca dane z formularza na format JSON.

Pliki statyczne, takie jak CSS, skrypty TypeScript czy grafiki, są przechowywane w katalogu \texttt{public}, a ich ścieżka jest przekazywana do middleware \texttt{express.static()}.

\subsection{Dockeryzacja}
Aby utworzyć obraz Dockera, został stworzony plik \texttt{Dockerfile}, który zawiera instrukcje dotyczące budowy obrazu. Następnie za pomocą polecenia:
\begin{lstlisting}
docker build -t portfolio .
\end{lstlisting}
został utworzony obraz o nazwie \texttt{portfolio}. Po zbudowaniu obrazu, można uruchomić kontener za pomocą polecenia:
\begin{lstlisting}
docker run -p port:port -d portfolio
\end{lstlisting}

\subsection{Serverless}
Aby przekształcić aplikację w tryb serverless, zostały wykorzystane dwie biblioteki: \texttt{serverless} oraz \texttt{serverless-http}. Następnie zmodyfikowano kod w pliku \texttt{server.ts} oraz dodano plik konfiguracyjny \texttt{serverless.yml}.

Za pomocą polecenia:
\begin{lstlisting}
serverless start-offline
\end{lstlisting}
można uruchomić aplikację w trybie offline, co pozwala na testowanie aplikacji bez potrzeby wdrażania jej na platformę chmurową. Problemem, który napotkałem podczas uruchamiania aplikacji w trybie offline, był brak dostępu do plików statycznych, tj. obrazów. Pozostałe pliki typu \texttt{js} oraz \texttt{css} w folderze \texttt{public} były dostępne. Mimo że pliki z obrazami były poprawnie przekazywane, to nie były wyświetlane. Żeby naprawić ten problem, trzeba było poprawić plik \texttt{serverless.yml}, dodając odpowiednie include'y i akceptowalne typy binarne mediów.

\subsection{Wdrożenie aplikacji na Vercel}
Aby wdrożyć aplikację na Vercel, należy zainstalować CLI Vercel oraz utworzyć konto na platformie. Następnie należy zalogować się do swojego konta za pomocą polecenia:
\begin{lstlisting}
vercel login
\end{lstlisting}
Po zalogowaniu się można wdrożyć aplikację za pomocą polecenia:
\begin{lstlisting}
vercel
\end{lstlisting}
Wdrożenie aplikacji przebiegło bezproblemowo. Nie wystąpiły również żadne problemy z ładowaniem treści strony, tj. obrazów czy plików JSON. Vercel domyślnie serwuje pliki znajdujące się w folderze publicznym, więc tam nawet nie występował problem napotkany przy lokalnym testowaniu aplikacji.

\end{document}
